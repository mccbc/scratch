 \documentclass[onecolumn]{aastex63}
\usepackage{amsmath}
\usepackage{listings}
\usepackage{tensor}

\lstset{frame=tb,
  language=Python,
  showstringspaces=false,
  columns=flexible,
  basicstyle={\small\ttfamily},
  commentstyle={\small\ttfamily},
  breaklines=true,
  breakatwhitespace=true,
  tabsize=4
}

\usepackage[T1]{fontenc}
\newcommand{\vdag}{(v)^\dagger}
\newcommand\aastex{AAS\TeX}
\newcommand\latex{La\TeX}
\shortauthors{McClellan}
\graphicspath{{./}{figures/}}


\begin{document}

\title{Scratch Work}
\author{B. Connor McClellan}
\affiliation{University of Virginia}
\keywords{}

\setlength\parindent{0pt}


\section{Polar Coordinates}
The probability distributions for polar coordinates $r$ and $\theta$ are
\begin{equation}
    P(r)dr \propto rdr
\end{equation}

\begin{equation}
    P(\theta)d\theta \propto d\theta
\end{equation}

Their respective cumulative probability distributions are

\begin{equation}
    \xi = \frac{\int_0^rr'dr'}{\int_0^{R}rdr} = \frac{r^2}{R^2}
\end{equation}

\begin{equation}
    \zeta = \frac{\int_0^\theta d\theta'}{\int_0^{2\pi}d\theta'} = \frac{\theta}{2\pi}
\end{equation}

for two uniform-random numbers between 0 and 1, $\xi$ and $\zeta$. Solving for $r$ and $\theta$,

\begin{equation}
    r = R\sqrt{\xi}
\end{equation}

\begin{equation}
    \theta = 2\pi \zeta
\end{equation}

The transformation matrix between spherical polar coordinates and Cartesian coordinates is $\Lambda \indices{^\alpha_{\bar{\alpha}}}$, where $\bar{\alpha}$ corresponds to polar coordinates and $\alpha$ corresponds to Cartesian.

\begin{equation}
\Lambda \indices{^\alpha_{\bar{\alpha}}} = 
\begin{bmatrix}
    \partial x / \partial r & \partial x / \partial \theta & \partial x / \partial \phi\\
    \partial y / \partial r & \partial y / \partial \theta & \partial y / \partial \phi\\
    \partial z / \partial r & \partial z / \partial \theta & \partial z / \partial \phi
\end{bmatrix}
=
\begin{bmatrix}
    \sin{\theta} \cos{\phi} & r \cos{\theta} \cos{\phi} & -r \sin{\theta} \sin{\phi}\\
    \sin{\theta} \sin{\phi} & r \cos{\theta} \sin{\phi} & r \sin{\theta} \cos{\phi}\\
    \cos{\theta} & -r \sin{\theta} & 0
\end{bmatrix}
\end{equation}

A vector in spherical coordinates can be transformed to Cartesian coordinates with

\begin{equation*}
    x^\alpha = \Lambda \indices{^\alpha_{\bar{\alpha}}} x^\bar{\alpha}
\end{equation*}

The $-\hat{z}$ unit vector in spherical polar coordinates is

\begin{equation}
    \begin{bmatrix}
        -\cos{\theta} \\
        \sin{\theta}\\
        0
    \end{bmatrix}
\end{equation}

Transforming to Cartesian, we obtain

\begin{equation}
    -\hat{z} = 
    \begin{bmatrix}
        \sin{\theta} \cos{\phi} & r \cos{\theta} \cos{\phi} & -r \sin{\theta} \sin{\phi}\\
        \sin{\theta} \sin{\phi} & r \cos{\theta} \sin{\phi} & r \sin{\theta} \cos{\phi}\\
        \cos{\theta} & -r \sin{\theta} & 0
    \end{bmatrix}
    \begin{bmatrix}
        -\cos{\theta} \\
        \sin{\theta}\\
        0
    \end{bmatrix}
    =
    \
\end{equation}

\end{document}