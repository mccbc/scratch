 \documentclass[onecolumn]{aastex63}
\usepackage{amsmath}
\usepackage{listings}
\usepackage{tensor}

\lstset{frame=tb,
  language=Python,
  showstringspaces=false,
  columns=flexible,
  basicstyle={\small\ttfamily},
  commentstyle={\small\ttfamily},
  breaklines=true,
  breakatwhitespace=true,
  tabsize=4
}

\usepackage[T1]{fontenc}
\newcommand{\vdag}{(v)^\dagger}
\newcommand\aastex{AAS\TeX}
\newcommand\latex{La\TeX}
\shortauthors{McClellan}
\graphicspath{{./}{figures/}}


\begin{document}

\title{Scratch Work}
\author{B. Connor McClellan}
\affiliation{University of Virginia}
\keywords{}

\setlength\parindent{0pt}

\ifx
\section{Polar Coordinates}
The probability distributions for polar coordinates $r$ and $\theta$ are
\begin{equation}
    P(r)dr \propto rdr
\end{equation}

\begin{equation}
    P(\theta)d\theta \propto d\theta
\end{equation}

Their respective cumulative probability distributions are

\begin{equation}
    \xi = \frac{\int_0^rr'dr'}{\int_0^{R}rdr} = \frac{r^2}{R^2}
\end{equation}

\begin{equation}
    \zeta = \frac{\int_0^\theta d\theta'}{\int_0^{2\pi}d\theta'} = \frac{\theta}{2\pi}
\end{equation}

for two uniform-random numbers between 0 and 1, $\xi$ and $\zeta$. Solving for $r$ and $\theta$,

\begin{equation}
    r = R\sqrt{\xi}
\end{equation}

\begin{equation}
    \theta = 2\pi \zeta
\end{equation}

The transformation matrix between spherical polar coordinates and Cartesian coordinates is $\Lambda \indices{^\alpha_{\bar{\alpha}}}$, where $\bar{\alpha}$ corresponds to polar coordinates and $\alpha$ corresponds to Cartesian.

\begin{equation}
\Lambda \indices{^\alpha_{\bar{\alpha}}} = 
\begin{bmatrix}
    \partial x / \partial r & \partial x / \partial \theta & \partial x / \partial \phi\\
    \partial y / \partial r & \partial y / \partial \theta & \partial y / \partial \phi\\
    \partial z / \partial r & \partial z / \partial \theta & \partial z / \partial \phi
\end{bmatrix}
=
\begin{bmatrix}
    \sin{\theta} \cos{\phi} & r \cos{\theta} \cos{\phi} & -r \sin{\theta} \sin{\phi}\\
    \sin{\theta} \sin{\phi} & r \cos{\theta} \sin{\phi} & r \sin{\theta} \cos{\phi}\\
    \cos{\theta} & -r \sin{\theta} & 0
\end{bmatrix}
\end{equation}

A vector in spherical coordinates can be transformed to Cartesian coordinates with

\begin{equation*}
    x^\alpha = \Lambda \indices{^\alpha_{\bar{\alpha}}} x^\bar{\alpha}
\end{equation*}

The $-\hat{z}$ unit vector in spherical polar coordinates is

\begin{equation}
    \begin{bmatrix}
        -\cos{\theta} \\
        \sin{\theta}\\
        0
    \end{bmatrix}
\end{equation}

Transforming to Cartesian, we obtain

\begin{equation}
    -\hat{z} = 
    \begin{bmatrix}
        \sin{\theta} \cos{\phi} & r \cos{\theta} \cos{\phi} & -r \sin{\theta} \sin{\phi}\\
        \sin{\theta} \sin{\phi} & r \cos{\theta} \sin{\phi} & r \sin{\theta} \cos{\phi}\\
        \cos{\theta} & -r \sin{\theta} & 0
    \end{bmatrix}
    \begin{bmatrix}
        -\cos{\theta} \\
        \sin{\theta}\\
        0
    \end{bmatrix}
    =
    \
\end{equation}
\fi

\section{Closed-Form Solutions for Ly$\alpha$ Resonant Scattering}

The flux at the surface of the sphere is 

\begin{equation}\label{eq:H0surf}
H_0(R,\nu)  =  \left( \frac{1}{3k\phi} \right)
\left( \frac{\sqrt{6}L}{32\pi \Delta} \right)
\left( \frac{1}{R^3} \right)
\left( 
\frac{ 1 }{ \cosh \left[ \frac{\pi \Delta}{k R} (\sigma - \sigma_{\rm s}) \right] +1 }
\right)
\end{equation}

where

\begin{equation} \label{delta}
    \Delta = \frac{kR}{\sqrt{\pi}\tau_0}
\end{equation}

Equation C17 in \citet{2006ApJ...649...14D} reads

\begin{equation} \label{dijkstra}
    J(x) = \frac{\sqrt{\pi}}{\sqrt{24}a\tau_0}\left(\frac{x^2}{1 + \cosh{\sqrt{2\pi^3/27}(|x^3|/a\tau_0)}}\right)
\end{equation}

where $x$ is related to our frequency variable $\sigma$ by 

\begin{equation} \label{sigma}
    \sigma = \frac{\sqrt{2\pi^2/27}}{a} x^3
\end{equation}

The Voigt function line profile is defined by

\begin{equation} \label{lineprofile}
    \phi = \frac{a}{\pi\Delta x^2}
\end{equation}

Eq. \ref{dijkstra} is rewritten in our notation to show that it does not satisfy the necessary boundary condition for the two-stream approximation:

\begin{equation}
    J = \sqrt{3} H
\end{equation}

We change variables and make the necessary substitutions according to Eqs. \ref{delta} and \ref{sigma} as follows.

\begin{equation}
\begin{split}
    J(x) &= \frac{\sqrt{\pi}}{\sqrt{24}a\tau_0}\left(\frac{x^2}{1 + \cosh{\sqrt{2\pi^3/27}(|x^3|/a\tau_0)}}\right)\\
    J(\sigma) &= \frac{\sqrt{\pi}}{\sqrt{24}a\tau_0}\left(\frac{a/\pi \Delta \phi}{1 + \cosh{\sqrt{2\pi^3/27}(|a \sigma / \sqrt{2\pi^2/27}|/a\tau_0)}}\right)\\
    J(\sigma) &= \frac{1}{\sqrt{24}}\frac{1}{\tau_0}\frac{1}{\sqrt{\pi}\Delta \phi}\left(\frac{1}{1 + \cosh{\left[\frac{\sqrt{\pi}}{\tau_0}|\sigma|\right]}}\right)\\
    J(\sigma) &=
    \frac{1}{\sqrt{24}}\frac{\sqrt{\pi} \Delta}{kR}\frac{1}{\sqrt{\pi}\Delta \phi}\left(\frac{1}{1 + \cosh{\left[\frac{\pi \Delta}{kR}|\sigma|\right]}}\right)\\
    J(\sigma) &=
    \left(\frac{1}{3k\phi}\right) \left(\frac{1}{2\sqrt{6}}\right)\left(\frac{1}{R}\right)\left(\frac{1}{1 + \cosh{\left[\frac{\pi \Delta}{kR}|\sigma|\right]}}\right)\\
\end{split}
\end{equation}

\citet{2006ApJ...649...14D} have multiplied their Eq. C17 by $4\pi R^2$ to obtain the total emerging flux density at the surface. Thus, dividing by the surface area of the sphere, we obtain

\begin{equation}
    J(\sigma) =
    \left(\frac{1}{3k\phi}\right) \left(\frac{1}{8\sqrt{6}\pi}\right)\left(\frac{1}{R^3}\right)\left(\frac{1}{1 + \cosh{\left[\frac{\pi \Delta}{kR}|\sigma|\right]}}\right)\\
\end{equation}

The missing term necessary to satisfy the BC is $\sqrt{3} \cdot 6L/(4\Delta)$, which would yield

\begin{equation}
    J(\sigma) = \sqrt{3}
    \left(\frac{1}{3k\phi}\right) \left(\frac{\sqrt{6}L}{32\pi\Delta}\right)\left(\frac{1}{R^3}\right)\left(\frac{1}{1 + \cosh{\left[\frac{\pi \Delta}{kR}|\sigma|\right]}}\right) = \sqrt{3} H\\
\end{equation}




\bibliography{bibliography}{}
\bibliographystyle{aasjournal}

\end{document}
