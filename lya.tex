 \documentclass[onecolumn]{aastex63}
\usepackage{amsmath}
\usepackage{listings}
\usepackage{tensor}

\lstset{frame=tb,
  language=Python,
  showstringspaces=false,
  columns=flexible,
  basicstyle={\small\ttfamily},
  commentstyle={\small\ttfamily},
  breaklines=true,
  breakatwhitespace=true,
  tabsize=4
}

\usepackage[T1]{fontenc}
\newcommand{\vdag}{(v)^\dagger}
\newcommand\aastex{AAS\TeX}
\newcommand\latex{La\TeX}
\shortauthors{McClellan}
\graphicspath{{./}{figures/}}


\begin{document}

\title{Scratch Work}
\author{B. Connor McClellan}
\affiliation{University of Virginia}
\keywords{}

\setlength\parindent{0pt}

\section{Closed-Form Solutions for Ly$\alpha$ Resonant Scattering}

$x$ is related to the frequency variable $\sigma$ by 

\begin{equation} \label{sigma}
    \sigma = \frac{\sqrt{2\pi/27}}{a} x^3
\end{equation}

The Voigt function line profile is defined as 

\begin{equation} \label{lineprofile}
    \phi = \frac{a}{\sqrt{\pi} x^2}
\end{equation}

and

\begin{equation} \label{tau}
    R\kappa_0 = \tau_0
\end{equation}

Equation C17 in \citet{2006ApJ...649...14D} reads

\begin{equation} \label{dijkstra}
    J(x) = \frac{\sqrt{\pi}}{\sqrt{24}a\tau_0}\left(\frac{x^2}{1 + \cosh{\sqrt{2\pi^3/27}(|x^3|/a\tau_0)}}\right)
\end{equation}

This will be checked against the boundary condition

\begin{equation} \label{bc}
    J = \sqrt{3} H
\end{equation}

by evaluating the surface flux derived from their Eq. C12. This equation reads 

\begin{equation} \label{c12}
    J(r, \sigma) = \frac{\sqrt{6}}{16 \pi^2 R} \frac{\kappa_0}{rr_s} \sum_{n=1}^{\infty}\sin(\lambda_n r) \sin(\lambda_n r_s) \frac{\exp{(-\lambda_n |\sigma|/\kappa_0)}}{\lambda_n}
\end{equation}

with

\begin{equation} \label{lambdan}
    \lambda_n = \frac{n\pi}{R}
\end{equation}

at lowest order. To find the flux, we calculate the following derivative.

\begin{equation}
    \begin{split}
    H(r, \sigma) &= - \frac{1}{3\kappa_0 \phi}\frac{\partial J}{\partial r}\\
    &= - \frac{1}{3\kappa_0 \phi}\frac{\partial}{\partial r}\left(\frac{\sqrt{6}}{16 \pi^2 R} \frac{\kappa_0}{rr_s} \sum_{n=1}^{\infty}\sin(\lambda_n r) \sin(\lambda_n r_s) \frac{\exp{(-\lambda_n |\sigma|/\kappa_0)}}{\lambda_n}\right)
    \end{split}
\end{equation}

Applying the product rule within the sum, we obtain

\begin{equation}
    H(r, \sigma) = - \frac{1}{3 \phi} \frac{\sqrt{6}}{16 \pi^2 R} \frac{1}{r_s} \sum_{n=1}^{\infty} \sin(\lambda_n r_s) \frac{\exp{(-\lambda_n |\sigma|/\kappa_0)}}{\lambda_n} \left[ \frac{-\sin(\lambda_n r)}{r^2} + \frac{\lambda_n \cos(\lambda_n r)}{r}\right]
\end{equation}

At this point, the expression is evaluated at the surface, $r=R$. Eq. \ref{lambdan} is substituted in for $\lambda_n$.

\begin{equation}
    H(\sigma) = - \frac{1}{3 \phi} \frac{\sqrt{6}}{16 \pi^2 R} \frac{1}{r_s} \sum_{n=1}^{\infty} \sin{\left(\frac{n\pi r_s}{R}\right)} \frac{\exp{(-n \pi |\sigma|/R\kappa_0)}}{\left(\frac{n\pi}{R}\right)} \left[ \frac{-\sin(n \pi)}{R^2} + \frac{n \pi \cos(n \pi)}{R^2}\right]
\end{equation}

The $\sin(n\pi)$ terms evaluate to zero, and the $\cos(n\pi)$ terms are equivalent to $(-1)^n$. We now have

\begin{equation}
    H(\sigma) = - \frac{1}{3 \phi} \frac{\sqrt{6}}{16 \pi^2 R^2} \frac{1}{r_s} \sum_{n=1}^{\infty} (-1)^n \sin{\left(\frac{n\pi r_s}{R}\right)} \exp{(-n \pi |\sigma|/R\kappa_0)}
\end{equation}

Writing out the $\sin$ and $\exp$ terms, we are able to combine them into two separate factors raised to the power $n$, allowing the sum to be written as two geometric series.

\begin{equation}
    \sin{x} = \frac{e^{ix}-e^{-ix}}{2i}
\end{equation}

\begin{equation}
    \begin{split}
        H(\sigma) &= - \frac{1}{3 \phi} \frac{\sqrt{6}}{16 \pi^2 R^2} \frac{1}{r_s} \sum_{n=1}^{\infty} (-1)^n \frac{\exp{(in\pi r_s / R)}-\exp{(-in\pi r_s/R})}{2i} \exp{(-n \pi |\sigma|/R\kappa_0)} \\
        &= - \frac{1}{3 \phi} \frac{\sqrt{6}}{32 i \pi^2 R^2} \frac{1}{r_s} \sum_{n=1}^{\infty} (-1)^n \left[ \exp{\left(\frac{-\pi |\sigma|}{R\kappa_0} + \frac{i\pi r_s}{R}\right)^n}-\exp{\left(\frac{-\pi |\sigma|}{R\kappa_0} - \frac{i\pi r_s}{R}\right)^n}\right]
    \end{split}
\end{equation}

Using the geometric series

\begin{equation}
    \sum_{n=1}^{\infty}(-1)^nx^n = \frac{-x}{1+x}, \hspace{0.2cm} |x| < 1
\end{equation}

the sum can be evaluated to obtain

\begin{equation}
    H(\sigma) = - \frac{1}{3 \phi} \frac{\sqrt{6}}{32 i \pi^2 R^2} \frac{1}{r_s} \left[\frac{-\exp{\left(\frac{-\pi |\sigma|}{R\kappa_0} + \frac{i\pi r_s}{R}\right)}}{1 + \exp{\left(\frac{-\pi |\sigma|}{R\kappa_0} + \frac{i\pi r_s}{R}\right)}}-\frac{-\exp{\left(\frac{-\pi |\sigma|}{R\kappa_0} - \frac{i\pi r_s}{R}\right)}}{1 + \exp{\left(\frac{-\pi |\sigma|}{R\kappa_0} - \frac{i\pi r_s}{R}\right)}}\right]
\end{equation}

Dividing top and bottom of each fraction through by its numerator, this simplifies to

\begin{equation} \label{before_trig}
    H(\sigma) = \frac{1}{3 \phi} \frac{\sqrt{6}}{32 i \pi^2 R^2} \frac{1}{r_s} \left[\frac{1}{1 + \exp{\left(\frac{\pi |\sigma|}{R\kappa_0} - \frac{i\pi r_s}{R}\right)}}-\frac{1}{1 + \exp{\left(\frac{\pi |\sigma|}{R\kappa_0} + \frac{i\pi r_s}{R}\right)}}\right]
\end{equation}

Let us rewrite the terms in brackets as trigonometric functions, using the following variables as their arguments.

\begin{equation}
    \alpha = \frac{\pi |\sigma|}{R\kappa_0}; \hspace{0.4cm} \beta = \frac{\pi r_s}{R}
\end{equation}

\begin{equation}
    \begin{split}
    \frac{1}{1+e^{\alpha - i\beta}} - \frac{1}{1+e^{\alpha + i\beta}} &= \frac{1+e^{\alpha + i\beta} - 1 - e^{\alpha - i\beta}}{\left(1+e^{\alpha - i\beta}\right)\left(1+e^{\alpha + i\beta}\right)} \\
    &= \frac{e^{\alpha}\left(e^{i\beta} - e^{- i\beta}\right)}{\left(1+e^{\alpha - i\beta}\right)\left(1+e^{\alpha + i\beta}\right)} \\
    &= \frac{e^{\alpha}}{1 + e^{\alpha - i\beta} + e^{\alpha + i\beta} + e^{2\alpha}} 2i \sin{\beta} \\
    &= \frac{2i \sin{\beta}}{e^{i\beta} + e^{- i\beta} + e^{\alpha} + e^{-\alpha}} \\
    &= \frac{i \sin{\beta}}{\cos{\beta} + \cosh{\alpha}}
    \end{split}
\end{equation}

With this, Eq. \ref{before_trig} becomes

\begin{equation} \label{after_trig}
    H(\sigma) = \frac{1}{3 \phi} \frac{\sqrt{6}}{32 \pi^2 R^2} \frac{1}{r_s} \left(\frac{\sin{\left(\frac{\pi r_s}{R}\right)}}{\cos{\left(\frac{\pi r_s}{R}\right)} + \cosh{\left(\frac{\pi |\sigma|}{R\kappa_0}\right)}}\right)
\end{equation}

We now take $r_s \rightarrow 0$, taking advantage of the approximation $\sin(x) \approx x$ for small $x$.

\begin{equation}
    H(\sigma) = \frac{1}{3 \phi} \frac{\sqrt{6}}{32 \pi R^3} \left(\frac{1}{1 + \cosh{\left(\frac{\pi |\sigma|}{R\kappa_0}\right)}}\right)
\end{equation}

Substituting in Eq. \ref{sigma} for $\sigma$, Eq. \ref{lineprofile} for $\phi$, Eq. \ref{tau} for $R \kappa_0$, and multiplying through by $4\pi R^2$ to obtain the total flux at the surface of the sphere, we are left with

\begin{equation} \label{final}
    H(x) = \frac{\sqrt{\pi}}{2\sqrt{24}aR} \left(\frac{x^2}{1 + \cosh{\left(\sqrt{2\pi^3/27}|x^3|/a\tau_0\right)}}\right)
\end{equation}

Notably, there appears to be a surplus factor of $\kappa_0$ in Eq. \ref{dijkstra}. If it were present in Eq. \ref{final}, it would combine with the $R$ in the denominator to form another factor of $\tau_0$ outside. In this case, the two equations would be related by the expression $J(x) = 2 H(x)$, suggesting that the boundary condition (Eq. \ref{bc}) is not satisfied.


\bibliography{bibliography}{}
\bibliographystyle{aasjournal}

\end{document}
